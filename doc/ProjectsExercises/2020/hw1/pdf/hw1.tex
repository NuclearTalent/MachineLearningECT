%%
%% Automatically generated file from DocOnce source
%% (https://github.com/hplgit/doconce/)
%%
%%


%-------------------- begin preamble ----------------------

\documentclass[%
oneside,                 % oneside: electronic viewing, twoside: printing
final,                   % draft: marks overfull hboxes, figures with paths
10pt]{article}

\listfiles               %  print all files needed to compile this document

\usepackage{relsize,makeidx,color,setspace,amsmath,amsfonts,amssymb}
\usepackage[table]{xcolor}
\usepackage{bm,ltablex,microtype}

\usepackage[pdftex]{graphicx}

% Tools for marking corrections
\usepackage{soul}
\newcommand{\replace}[2]{{\color{red}\text{\st{#1} #2}}}
\newcommand{\remove}[1]{{\color{red}\st{#1}}}

\usepackage{fancyvrb} % packages needed for verbatim environments
\usepackage{minted}
\usemintedstyle{default}

\usepackage[T1]{fontenc}
%\usepackage[latin1]{inputenc}
\usepackage{ucs}
\usepackage[utf8x]{inputenc}

\usepackage{lmodern}         % Latin Modern fonts derived from Computer Modern

% Hyperlinks in PDF:
\definecolor{linkcolor}{rgb}{0,0,0.4}
\usepackage{hyperref}
\hypersetup{
    breaklinks=true,
    colorlinks=true,
    linkcolor=linkcolor,
    urlcolor=linkcolor,
    citecolor=black,
    filecolor=black,
    %filecolor=blue,
    pdfmenubar=true,
    pdftoolbar=true,
    bookmarksdepth=3   % Uncomment (and tweak) for PDF bookmarks with more levels than the TOC
    }
%\hyperbaseurl{}   % hyperlinks are relative to this root

\setcounter{tocdepth}{2}  % levels in table of contents

% newcommands for typesetting inline (doconce) comments
\newcommand{\shortinlinecomment}[3]{{\color{red}{\bf #1}: #2}}
\newcommand{\longinlinecomment}[3]{{\color{red}{\bf #1}: #2}}

\usepackage[framemethod=TikZ]{mdframed}

% --- begin definitions of admonition environments ---

% --- end of definitions of admonition environments ---

% prevent orhpans and widows
\clubpenalty = 10000
\widowpenalty = 10000

\newenvironment{doconceexercise}{}{}
\newcounter{doconceexercisecounter}


% ------ header in subexercises ------
%\newcommand{\subex}[1]{\paragraph{#1}}
%\newcommand{\subex}[1]{\par\vspace{1.7mm}\noindent{\bf #1}\ \ }
\makeatletter
% 1.5ex is the spacing above the header, 0.5em the spacing after subex title
\newcommand\subex{\@startsection*{paragraph}{4}{\z@}%
                  {1.5ex\@plus1ex \@minus.2ex}%
                  {-0.5em}%
                  {\normalfont\normalsize\bfseries}}
\makeatother


% --- end of standard preamble for documents ---


% insert custom LaTeX commands...

\raggedbottom
\makeindex
\usepackage[totoc]{idxlayout}   % for index in the toc
\usepackage[nottoc]{tocbibind}  % for references/bibliography in the toc

%-------------------- end preamble ----------------------

\begin{document}

% matching end for #ifdef PREAMBLE

\newcommand{\exercisesection}[1]{\subsection*{#1}}


% ------------------- main content ----------------------



% ----------------- title -------------------------

\thispagestyle{empty}

\begin{center}
{\LARGE\bf
\begin{spacing}{1.25}
First and second days: Exercise set 1
\end{spacing}
}
\end{center}

% ----------------- author(s) -------------------------

\begin{center}
{\bf Data Analysis and Machine Learning for Nuclear Physics${}^{}$} \\ [0mm]
\end{center}

\begin{center}
% List of all institutions:
\end{center}
    
% ----------------- end author(s) -------------------------

% --- begin date ---
\begin{center}
Jul 1, 2020
\end{center}
% --- end date ---

\vspace{1cm}


\subsection*{Day one and two  exercises}




% --- begin exercise ---
\begin{doconceexercise}
\refstepcounter{doconceexercisecounter}

\exercisesection*{Exercise \thedoconceexercisecounter: Getting started}


The first exercise here is of a mere technical art. We want you to have 
\begin{itemize}
\item git as a version control software and to establish a user account on a provider like GitHub. Other providers like GitLab etc are equally fine. 

\item Install various Python packages
\end{itemize}

\noindent
We will make extensive use of Python as programming language and its
myriad of available libraries.  You will find
IPython/Jupyter notebooks invaluable in your work.  You can run \textbf{R}
codes in the Jupyter/IPython notebooks, with the immediate benefit of
visualizing your data. You can also use compiled languages like C++,
Rust, Fortran etc if you prefer. The focus in these lectures will be
on Python.

If you have Python installed (we recommend Python3) and you feel
pretty familiar with installing different packages, we recommend that
you install the following Python packages via \textbf{pip} as 

\begin{enumerate}
\item pip install numpy scipy matplotlib ipython scikit-learn sympy pandas pillow 
\end{enumerate}

\noindent
For \textbf{Tensorflow}, we recommend following the instructions in the text of 
\href{{http://shop.oreilly.com/product/0636920052289.do}}{Aurelien Geron, Hands‑On Machine Learning with Scikit‑Learn and TensorFlow, O'Reilly}

We will come back to \textbf{tensorflow} later. 

For Python3, replace \textbf{pip} with \textbf{pip3}.

For OSX users we recommend, after having installed Xcode, to
install \textbf{brew}. Brew allows for a seamless installation of additional
software via for example 

\begin{enumerate}
\item brew install python3
\end{enumerate}

\noindent
For Linux users, with its variety of distributions like for example the widely popular Ubuntu distribution,
you can use \textbf{pip} as well and simply install Python as 

\begin{enumerate}
\item sudo apt-get install python3  (or python for Python2.7)
\end{enumerate}

\noindent
If you don't want to perform these operations separately and venture
into the hassle of exploring how to set up dependencies and paths, we
recommend two widely used distrubutions which set up all relevant
dependencies for Python, namely 

\begin{itemize}
\item \href{{https://docs.anaconda.com/}}{Anaconda}, 
\end{itemize}

\noindent
which is an open source
distribution of the Python and R programming languages for large-scale
data processing, predictive analytics, and scientific computing, that
aims to simplify package management and deployment. Package versions
are managed by the package management system \textbf{conda}. 

\begin{itemize}
\item \href{{https://www.enthought.com/product/canopy/}}{Enthought canopy} 
\end{itemize}

\noindent
is a Python
distribution for scientific and analytic computing distribution and
analysis environment, available for free and under a commercial
license.

We recommend using \textbf{Anaconda}.

\end{doconceexercise}
% --- end exercise ---




% --- begin exercise ---
\begin{doconceexercise}
\refstepcounter{doconceexercisecounter}

\exercisesection*{Exercise \thedoconceexercisecounter: Our first Python encounter}


This exercise has as its aim to write a small program which reads in data from a \textbf{csv} file on the equation of state for dense nuclear matter. The file is localized at \href{{https://github.com/mhjensen/MachineLearningMSU-FRIB2020/blob/master/doc/pub/Regression/ipynb/datafiles/EoS.csv}}{\nolinkurl{https://github.com/mhjensen/MachineLearningMSU-FRIB2020/blob/master/doc/pub/Regression/ipynb/datafiles/EoS.csv}}. Thereafter you will have to set up the design matrix $\bm{X}$ for the  $n$
datapoints and a polynomial of degree $3$. The steps are:
\begin{itemize}
\item Write a Python code which reads the in the above mentioned file.

\item Use for example \textbf{pandas} to order your data and find out how many data points there are.

\item Set thereafter up the design matrix with dimensionality $n\times p$ where $p=4$ and where you have defined a polynomial of degree $p-1=3$. Print the matrix and check that the numbers are correct. 
\end{itemize}

\noindent
We recommend looking at the examples in the \href{{https://compphysics.github.io/MachineLearning/doc/pub/Regression/html/Regression-bs.html}}{regression slides}.


% --- begin solution of exercise ---
\paragraph{Solution.}
\begin{minted}[fontsize=\fontsize{9pt}{9pt},linenos=false,mathescape,baselinestretch=1.0,fontfamily=tt,xleftmargin=7mm]{python}
import os
import numpy as np
import pandas as pd
import matplotlib.pyplot as plt
from sklearn.model_selection import train_test_split
# Where to save the figures and data files
PROJECT_ROOT_DIR = "Results"
FIGURE_ID = "Results/FigureFiles"
DATA_ID = "DataFiles/"

if not os.path.exists(PROJECT_ROOT_DIR):
    os.mkdir(PROJECT_ROOT_DIR)

if not os.path.exists(FIGURE_ID):
    os.makedirs(FIGURE_ID)

if not os.path.exists(DATA_ID):
    os.makedirs(DATA_ID)

def image_path(fig_id):
    return os.path.join(FIGURE_ID, fig_id)

def data_path(dat_id):
    return os.path.join(DATA_ID, dat_id)

def save_fig(fig_id):
    plt.savefig(image_path(fig_id) + ".png", format='png')

def R2(y_data, y_model):
    return 1 - np.sum((y_data - y_model) ** 2) / np.sum((y_data - np.mean(y_data)) ** 2)
def MSE(y_data,y_model):
    n = np.size(y_model)
    return np.sum((y_data-y_model)**2)/n

infile = open(data_path("EoS.csv"),'r')

# Read the EoS data as  csv file and organized into two arrays with density and energies
EoS = pd.read_csv(infile, names=('Density', 'Energy'))
EoS['Energy'] = pd.to_numeric(EoS['Energy'], errors='coerce')
EoS = EoS.dropna()
Energies = EoS['Energy']
Density = EoS['Density']
#  The design matrix now as function of various polytrops
X = np.zeros((len(Density),5))
X[:,0] = 1
X[:,1] = Density**(2.0/3.0)
X[:,2] = Density
X[:,3] = Density**(4.0/3.0)
X[:,4] = Density**(5.0/3.0)
# We split the data in test and training data
X_train, X_test, y_train, y_test = train_test_split(X, Energies, test_size=0.2)
# matrix inversion to find beta
beta = np.linalg.inv(X_train.T @ X_train) @ X_train.T @ y_train
# and then make the prediction
ytilde = X_train @ beta
print("Training R2")
print(R2(y_train,ytilde))
print("Training MSE")
print(MSE(y_train,ytilde))
ypredict = X_test @ beta
print("Test R2")
print(R2(y_test,ypredict))
print("Test MSE")
print(MSE(y_test,ypredict))
\end{minted}

% --- end solution of exercise ---

\end{doconceexercise}
% --- end exercise ---




% --- begin exercise ---
\begin{doconceexercise}
\refstepcounter{doconceexercisecounter}

\exercisesection*{Exercise \thedoconceexercisecounter: making your own data and exploring scikit-learn}


We will generate our own dataset for a function $y(x)$ where $x \in [0,1]$ and defined by random numbers computed with the uniform distribution. The function $y$ is a quadratic polynomial in $x$ with added stochastic noise according to the normal distribution $\cal {N}(0,1)$.
The following simple Python instructions define our $x$ and $y$ values (with 100 data points).
\begin{minted}[fontsize=\fontsize{9pt}{9pt},linenos=false,mathescape,baselinestretch=1.0,fontfamily=tt,xleftmargin=7mm]{python}
x = np.random.rand(100,1)
y = 2.0+5*x*x+0.1*np.random.randn(100,1)
\end{minted}

\begin{enumerate}
\item Write your own code (following the examples under the \href{{https://compphysics.github.io/MachineLearningECT/doc/pub/Day1/html/Day1-bs.html}}{regression slides}) for computing the parametrization of the data set fitting a second-order polynomial. 

\item Use thereafter \textbf{scikit-learn} (see again the examples in the regression slides) and compare with your own code.   

\item Using scikit-learn, compute also the mean square error, a risk metric corresponding to the expected value of the squared (quadratic) error defined as
\end{enumerate}

\noindent
\[ MSE(\hat{y},\hat{\tilde{y}}) = \frac{1}{n}
\sum_{i=0}^{n-1}(y_i-\tilde{y}_i)^2, 
\] 
and the $R^2$ score function.
If $\tilde{\hat{y}}_i$ is the predicted value of the $i-th$ sample and $y_i$ is the corresponding true value, then the score $R^2$ is defined as
\[
R^2(\hat{y}, \tilde{\hat{y}}) = 1 - \frac{\sum_{i=0}^{n - 1} (y_i - \tilde{y}_i)^2}{\sum_{i=0}^{n - 1} (y_i - \bar{y})^2},
\]
where we have defined the mean value  of $\hat{y}$ as
\[
\bar{y} =  \frac{1}{n} \sum_{i=0}^{n - 1} y_i.
\]
You can use the functionality included in scikit-learn. If you feel for it, you can use your own program and define functions which compute the above two functions. 
Discuss the meaning of these results. Try also to vary the coefficient in front of the added stochastic noise term and discuss the quality of the fits.


% --- begin solution of exercise ---
\paragraph{Solution.}
The code here is an example of where we define our own design matrix and fit parameters $\beta$.
\begin{minted}[fontsize=\fontsize{9pt}{9pt},linenos=false,mathescape,baselinestretch=1.0,fontfamily=tt,xleftmargin=7mm]{python}
import os
import numpy as np
import pandas as pd
import matplotlib.pyplot as plt
from sklearn.model_selection import train_test_split

def save_fig(fig_id):
    plt.savefig(image_path(fig_id) + ".png", format='png')

def R2(y_data, y_model):
    return 1 - np.sum((y_data - y_model) ** 2) / np.sum((y_data - np.mean(y_data)) ** 2)
def MSE(y_data,y_model):
    n = np.size(y_model)
    return np.sum((y_data-y_model)**2)/n

x = np.random.rand(100)
y = 2.0+5*x*x+0.1*np.random.randn(100)


#  The design matrix now as function of a given polynomial
X = np.zeros((len(x),3))
X[:,0] = 1.0
X[:,1] = x
X[:,2] = x**2
# We split the data in test and training data
X_train, X_test, y_train, y_test = train_test_split(X, y, test_size=0.2)
# matrix inversion to find beta
beta = np.linalg.inv(X_train.T @ X_train) @ X_train.T @ y_train
print(beta)
# and then make the prediction
ytilde = X_train @ beta
print("Training R2")
print(R2(y_train,ytilde))
print("Training MSE")
print(MSE(y_train,ytilde))
ypredict = X_test @ beta
print("Test R2")
print(R2(y_test,ypredict))
print("Test MSE")
print(MSE(y_test,ypredict))
\end{minted}

% --- end solution of exercise ---

\end{doconceexercise}
% --- end exercise ---




% --- begin exercise ---
\begin{doconceexercise}
\refstepcounter{doconceexercisecounter}

\exercisesection*{Exercise \thedoconceexercisecounter: mean values and variances in linear regression}


This exercise deals with various mean values ad variances in  linear regression method (here it may be useful to look up chapter 3, equation (3.8) of \href{{https://www.springer.com/gp/book/9780387848570}}{Trevor Hastie, Robert Tibshirani, Jerome H. Friedman, The Elements of Statistical Learning, Springer}).

The assumption we have made is 
that there exists a function $f(\bm{x})$ and  a normal distributed error $\bm{\varepsilon}\sim \mathcal{N}(0, \sigma^2)$
which describes our data
\[
\bm{y} = f(\bm{x})+\bm{\varepsilon}
\]

We then approximate this function with our model from the solution of the linear regression equations (ordinary least squares OLS), that is our
function $f$ is approximated by $\bm{\tilde{y}}$ where we minimized  $(\bm{y}-\bm{\tilde{y}})^2$, with
\[
\bm{\tilde{y}} = \bm{X}\bm{\beta}.
\]
The matrix $\bm{X}$ is the so-called design matrix.


\subex{a)}
Show that  the expectation value of $\bm{y}$ for a given element $i$ 
\begin{align*} 
\mathbb{E}(y_i) & =\mathbf{X}_{i, \ast} \, \beta, 
\end{align*} 
and that
its variance is 
\begin{align*} \mbox{Var}(y_i) & = \sigma^2.  
\end{align*}
Hence, $y_i \sim \mathcal{N}( \mathbf{X}_{i, \ast} \, \bm{\beta}, \sigma^2)$, that is $\bm{y}$ follows a normal distribution with 
mean value $\bm{X}\bm{\beta}$ and variance $\sigma^2$.


% --- begin solution of exercise ---
\paragraph{Solution.}
We can calculate the expectation value of $\bm{y}$ for a given element $i$ 
\begin{align*} 
\mathbb{E}(y_i) & =
\mathbb{E}(\mathbf{X}_{i, \ast} \, \bm{\beta}) + \mathbb{E}(\varepsilon_i)
\, \, \, = \, \, \, \mathbf{X}_{i, \ast} \, \beta, 
\end{align*} 
while
its variance is 
\begin{align*} \mbox{Var}(y_i) & = \mathbb{E} \{ [y_i
- \mathbb{E}(y_i)]^2 \} \, \, \, = \, \, \, \mathbb{E} ( y_i^2 ) -
[\mathbb{E}(y_i)]^2  \\  & = \mathbb{E} [ ( \mathbf{X}_{i, \ast} \,
\beta + \varepsilon_i )^2] - ( \mathbf{X}_{i, \ast} \, \bm{\beta})^2 \\ &
= \mathbb{E} [ ( \mathbf{X}_{i, \ast} \, \bm{\beta})^2 + 2 \varepsilon_i
\mathbf{X}_{i, \ast} \, \bm{\beta} + \varepsilon_i^2 ] - ( \mathbf{X}_{i,
\ast} \, \beta)^2 \\  & = ( \mathbf{X}_{i, \ast} \, \bm{\beta})^2 + 2
\mathbb{E}(\varepsilon_i) \mathbf{X}_{i, \ast} \, \bm{\beta} +
\mathbb{E}(\varepsilon_i^2 ) - ( \mathbf{X}_{i, \ast} \, \bm{\beta})^2 
\\ & = \mathbb{E}(\varepsilon_i^2 ) \, \, \, = \, \, \,
\mbox{Var}(\varepsilon_i) \, \, \, = \, \, \, \sigma^2.  
\end{align*}
Hence, $y_i \sim \mathcal{N}( \mathbf{X}_{i, \ast} \, \bm{\beta}, \sigma^2)$, that is $\bm{y}$ follows a normal distribution with 
mean value $\bm{X}\bm{\beta}$ and variance $\sigma^2$ (not be confused with the singular values of the SVD).

% --- end solution of exercise ---

\subex{b)}
With the OLS expressions for the parameters $\bm{\beta}$ show that
\[
\mathbb{E}(\bm{\beta}) = \bm{\beta}.
\]


% --- begin solution of exercise ---
\paragraph{Solution.}
\[
\mathbb{E}(\bm{\beta}) = \mathbb{E}[ (\mathbf{X}^{\top} \mathbf{X})^{-1}\mathbf{X}^{T} \mathbf{Y}]=(\mathbf{X}^{T} \mathbf{X})^{-1}\mathbf{X}^{T} \mathbb{E}[ \mathbf{Y}]=(\mathbf{X}^{T} \mathbf{X})^{-1} \mathbf{X}^{T}\mathbf{X}\bm{\beta}=\bm{\beta}.
\]
This means that the estimator of the regression parameters is unbiased.

% --- end solution of exercise ---

\subex{c)}
Show finally that the variance of $\bm{\beta}$ is
\begin{eqnarray*}
\mbox{Var}(\bm{\beta}) & = & \sigma^2 \, (\mathbf{X}^{T} \mathbf{X})^{-1}.
\end{eqnarray*}


% --- begin solution of exercise ---
\paragraph{Solution.}
The variance of $\bm{\beta}$ is
\begin{eqnarray*}
\mbox{Var}(\bm{\beta}) & = & \mathbb{E} \{ [\bm{\beta} - \mathbb{E}(\bm{\beta})] [\bm{\beta} - \mathbb{E}(\bm{\beta})]^{T} \}
\\
& = & \mathbb{E} \{ [(\mathbf{X}^{T} \mathbf{X})^{-1} \, \mathbf{X}^{T} \mathbf{Y} - \bm{\beta}] \, [(\mathbf{X}^{T} \mathbf{X})^{-1} \, \mathbf{X}^{T} \mathbf{Y} - \bm{\beta}]^{T} \}
\\
% & = & \mathbb{E} \{ [(\mathbf{X}^{T} \mathbf{X})^{-1} \, \mathbf{X}^{T} \mathbf{Y}] \, [(\mathbf{X}^{T} \mathbf{X})^{-1} \, \mathbf{X}^{T} \mathbf{Y}]^{T} \} - \bm{\beta} \, \bm{\beta}^{T}
% \\
% & = & \mathbb{E} \{ (\mathbf{X}^{T} \mathbf{X})^{-1} \, \mathbf{X}^{T} \mathbf{Y} \, \mathbf{Y}^{T} \, \mathbf{X} \, (\mathbf{X}^{T} \mathbf{X})^{-1}  \} - \bm{\beta} \, \bm{\beta}^{T}
% \\
& = & (\mathbf{X}^{T} \mathbf{X})^{-1} \, \mathbf{X}^{T} \, \mathbb{E} \{ \mathbf{Y} \, \mathbf{Y}^{T} \} \, \mathbf{X} \, (\mathbf{X}^{T} \mathbf{X})^{-1} - \bm{\beta} \, \bm{\beta}^{T}
\\
& = & (\mathbf{X}^{T} \mathbf{X})^{-1} \, \mathbf{X}^{T} \, \{ \mathbf{X} \, \bm{\beta} \, \bm{\beta}^{T} \,  \mathbf{X}^{T} + \sigma^2 \} \, \mathbf{X} \, (\mathbf{X}^{T} \mathbf{X})^{-1} - \bm{\beta} \, \bm{\beta}^{T}
% \\
% & = & (\mathbf{X}^T \mathbf{X})^{-1} \, \mathbf{X}^T \, \mathbf{X} \, \bm{\beta} \, \bm{\beta}^T \,  \mathbf{X}^T \, \mathbf{X} \, (\mathbf{X}^T % \mathbf{X})^{-1}
% \\
% & & + \, \, \sigma^2 \, (\mathbf{X}^T \mathbf{X})^{-1} \, \mathbf{X}^T  \, \mathbf{X} \, (\mathbf{X}^T \mathbf{X})^{-1} - \bm{\beta} \bm{\beta}^T
\\
& = & \bm{\beta} \, \bm{\beta}^{T}  + \sigma^2 \, (\mathbf{X}^{T} \mathbf{X})^{-1} - \bm{\beta} \, \bm{\beta}^{T}
\, \, \, = \, \, \, \sigma^2 \, (\mathbf{X}^{T} \mathbf{X})^{-1},
\end{eqnarray*}

where we have used  that $\mathbb{E} (\mathbf{Y} \mathbf{Y}^{T}) =
\mathbf{X} \, \bm{\beta} \, \bm{\beta}^{T} \, \mathbf{X}^{T} +
\sigma^2 \, \mathbf{I}_{nn}$. From $\mbox{Var}(\bm{\beta}) = \sigma^2
\, (\mathbf{X}^{T} \mathbf{X})^{-1}$, one obtains an estimate of the
variance of the estimate of the $j$-th regression coefficient:
$\bm{\sigma}^2 (\hat{\beta}_j ) = \bm{\sigma}^2 \sqrt{
[(\mathbf{X}^{T} \mathbf{X})^{-1}]_{jj} }$. This may be used to
construct a confidence interval for the estimates.


In a similar way, we can obtain analytical expressions for say the
expectation values of the parameters $\bm{\beta}$ and their variance
when we employ Ridge regression, allowing us again to define a confidence interval.

% --- end solution of exercise ---

\end{doconceexercise}
% --- end exercise ---




% --- begin exercise ---
\begin{doconceexercise}
\refstepcounter{doconceexercisecounter}

\exercisesection*{Exercise \thedoconceexercisecounter: Playing with nuclear masses}


Finally, try now to write your own code (you can use the example the nuclear masses in the lecture slides on Regression and Getting started from Day1, that reads in the nuclear masses and
compute the proton separation energies, the two-neutron and two-proton separation energies and finally the shell gaps for selected nuclei.

Finally, try to compute the $Q$-values for $\beta-$ decay for selected nuclei.


% --- begin solution of exercise ---
\paragraph{Solution.}
Let us study the $Q$ values associated with the removal of one or two nucleons from
a nucleus. These are conventionally defined in terms of the one-nucleon and two-nucleon
separation energies. With the functionality in \textbf{pandas}, two to three lines of code will allow us to plot the separation energies.
The neutron separation energy is defined as 

\[
S_n= -Q_n= BE(N,Z)-BE(N-1,Z),
\]
and the proton separation energy reads
\[
S_p= -Q_p= BE(N,Z)-BE(N,Z-1).
\]
The two-neutron separation energy is defined as
\[
S_{2n}= -Q_{2n}= BE(N,Z)-BE(N-2,Z),
\]
and  the two-proton separation energy is given by
\[
S_{2p}= -Q_{2p}= BE(N,Z)-BE(N,Z-2).
\]

Using say the neutron separation energies (alternatively the proton separation energies)
\[
S_n= -Q_n= BE(N,Z)-BE(N-1,Z),
\]
we can define the so-called energy gap for neutrons (or protons) as 
\[
\Delta S_n= BE(N,Z)-BE(N-1,Z)-\left(BE(N+1,Z)-BE(N,Z)\right),
\]
or 
\[
\Delta S_n= 2BE(N,Z)-BE(N-1,Z)-BE(N+1,Z).
\]
This quantity can in turn be used to determine which nuclei could be interpreted as  magic or not. 
For protons we would have 
\[
\Delta S_p= 2BE(N,Z)-BE(N,Z-1)-BE(N,Z+1).
\]

To calculate say the neutron separation we need to multiply our masses with the nucleon number $A$.
The example here is for the neutron separation energies for the oxygen isotopes.
Note the simple function we use to compute the neutron separation energies
\begin{minted}[fontsize=\fontsize{9pt}{9pt},linenos=false,mathescape,baselinestretch=1.0,fontfamily=tt,xleftmargin=7mm]{python}
# Her we pick the oyxgen isotopes
Nucleus = df.loc[lambda df: df.Z==8, :]
# drop cases with no number
Nucleus = Nucleus.dropna()
# Here we do the magic and obtain the neutron separation energies, one line of code!!
Nucleus['NeutronSeparationEnergies'] = Nucleus['Energies'].diff(+1)
\end{minted}

If we want another isotope we need simply to change the $Z$ value. For isotones, we fix simply the neutron number. Furthermore, if we wish to compute say the two-neutron separation energies of the oyxgen isotopes we need simply to write 
\begin{minted}[fontsize=\fontsize{9pt}{9pt},linenos=false,mathescape,baselinestretch=1.0,fontfamily=tt,xleftmargin=7mm]{python}
# Her we pick the oyxgen isotopes
Nucleus = df.loc[lambda df: df.Z==8, :]
# drop cases with no number
Nucleus = Nucleus.dropna()
# Here we do the magic and obtain the neutron separation energies, one line of code!!
Nucleus['NeutronSeparationEnergies'] = Nucleus['Energies'].diff(+2)
\end{minted}
Note the $+2$ in the function \textbf{diff(+2)}!  Easy, isn't it? It is easy to change to two-proton separation energies. The full example here is for the neutron separation energies.

\begin{minted}[fontsize=\fontsize{9pt}{9pt},linenos=false,mathescape,baselinestretch=1.0,fontfamily=tt,xleftmargin=7mm]{python}
# Common imports
import numpy as np
import pandas as pd
import matplotlib.pyplot as plt
import os
from pylab import plt, mpl
plt.style.use('seaborn')
mpl.rcParams['font.family'] = 'serif'


# Where to save the figures and data files
PROJECT_ROOT_DIR = "Results"
FIGURE_ID = "Results/FigureFiles"
DATA_ID = "DataFiles/"

if not os.path.exists(PROJECT_ROOT_DIR):
    os.mkdir(PROJECT_ROOT_DIR)

if not os.path.exists(FIGURE_ID):
    os.makedirs(FIGURE_ID)

if not os.path.exists(DATA_ID):
    os.makedirs(DATA_ID)

def image_path(fig_id):
    return os.path.join(FIGURE_ID, fig_id)

def data_path(dat_id):
    return os.path.join(DATA_ID, dat_id)

def save_fig(fig_id):
    plt.savefig(image_path(fig_id) + ".png", format='png')

infile = open(data_path("MassEval2016.dat"),'r')


# Read the experimental data with Pandas
Masses = pd.read_fwf(infile, usecols=(2,3,4,6,11),
              names=('N', 'Z', 'A', 'Element', 'Ebinding'),
              widths=(1,3,5,5,5,1,3,4,1,13,11,11,9,1,2,11,9,1,3,1,12,11,1),
              header=39,
              index_col=False)

# Extrapolated values are indicated by '#' in place of the decimal place, so
# the Ebinding column won't be numeric. Coerce to float and drop these entries.
Masses['Ebinding'] = pd.to_numeric(Masses['Ebinding'], errors='coerce')
Masses = Masses.dropna()
# Convert from keV to MeV.
Masses['Ebinding'] /= 1000
A = Masses['A']
Z = Masses['Z']
N = Masses['N']
Element = Masses['Element']
Energies = Masses['Ebinding']*A

df = pd.DataFrame({'A':A,'Z':Z, 'N':N,'Element':Element,'Energies':Energies})
# Her we pick the oyxgen isotopes
Nucleus = df.loc[lambda df: df.Z==8, :]
# drop cases with no number
Nucleus = Nucleus.dropna()
# Here we do the magic and obtain the neutron separation energies, one line of code!!
Nucleus['NeutronSeparationEnergies'] = Nucleus['Energies'].diff(+1)
print(Nucleus)
MakePlot([Nucleus.A], [Nucleus.NeutronSeparationEnergies], ['b'], ['Neutron Separation Energy'], ['$A$','$S_n$'])
save_fig('Nucleus')
plt.show()

\end{minted}

% --- end solution of exercise ---

\end{doconceexercise}
% --- end exercise ---


% !split
\subsection*{Expectation value and variance for $\bm{\beta}$}


We can also calculate the variance



% !split
\subsection*{Resampling methods}

% !split
\subsection*{Resampling methods, basic overview}

With all these analytical equations for both the OLS and Ridge
regression, we will now outline how to assess a given model. This will
lead us to a discussion of the so-called bias-variance tradeoff (see
below) and so-called resampling methods.

One of the quantities we have discussed as a way to measure errors is
the mean-squared error (MSE), mainly used for fitting of continuous
functions. Another choice is the absolute error.

In the discussions below we will focus on the MSE and in particular since we will split the data into test and training data,
we discuss the
\begin{enumerate}
\item prediction error or simply the \textbf{test error}, where we have a fixed training set and the test error is the MSE arising from the data reserved for testing. We discuss also the 

\item training error $\mathrm{Err_{Train}}$, which is the average loss over the training data.
\end{enumerate}

\noindent
As our model becomes more and more complex, more of the training data
tends to used. The training may thence adapt to more complicated
structures in the data. This may lead to a decrease in the bias (see
below for code example) and a slight increase of the variance for the
test error.  For a certain level of complexity the test error will
reach a minimum, before starting to increase again. The training error
reaches a saturation.

% !split
\subsection*{Train and test,  an example}
\begin{minted}[fontsize=\fontsize{9pt}{9pt},linenos=false,mathescape,baselinestretch=1.0,fontfamily=tt,xleftmargin=7mm]{python}
import matplotlib.pyplot as plt
import numpy as np
from sklearn.linear_model import LinearRegression, Ridge, Lasso
from sklearn.preprocessing import PolynomialFeatures
from sklearn.model_selection import train_test_split
from sklearn.pipeline import make_pipeline
from sklearn.utils import resample

np.random.seed(2018)
n = 50
maxdegree = 15

# Make data set.
x = np.linspace(-3, 3, n).reshape(-1, 1)
y = np.exp(-x**2) + 1.5 * np.exp(-(x-2)**2)+ np.random.normal(0, 0.1, x.shape)
TestError = np.zeros(maxdegree)
TrainError = np.zeros(maxdegree)
polydegree = np.zeros(maxdegree)
x_train, x_test, y_train, y_test = train_test_split(x, y, test_size=0.2)

for degree in range(maxdegree):
    model = make_pipeline(PolynomialFeatures(degree=degree), LinearRegression(fit_intercept=False))
    clf = model.fit(x_train,y_train)
    y_fit = clf.predict(x_train)
    y_pred = clf.predict(x_test) 
    polydegree[degree] = degree
    TestError[degree] = np.mean( np.mean((y_test - y_pred)**2) )
    TrainError[degree] = np.mean( np.mean((y_train - y_fit)**2) )

plt.plot(polydegree, TestError, label='Test Error')
plt.plot(polydegree, TrainError, label='Train Error')
plt.legend()
plt.show()
\end{minted}

% !split
\subsection*{Bringing back our Equation of State data}

\begin{minted}[fontsize=\fontsize{9pt}{9pt},linenos=false,mathescape,baselinestretch=1.0,fontfamily=tt,xleftmargin=7mm]{python}
# Common imports
import os
import numpy as np
import pandas as pd
import matplotlib.pyplot as plt
from sklearn.linear_model import LinearRegression, Ridge, Lasso
from sklearn.model_selection import train_test_split
from sklearn.utils import resample
from sklearn.metrics import mean_squared_error
# Where to save the figures and data files
PROJECT_ROOT_DIR = "Results"
FIGURE_ID = "Results/FigureFiles"
DATA_ID = "DataFiles/"

if not os.path.exists(PROJECT_ROOT_DIR):
    os.mkdir(PROJECT_ROOT_DIR)

if not os.path.exists(FIGURE_ID):
    os.makedirs(FIGURE_ID)

if not os.path.exists(DATA_ID):
    os.makedirs(DATA_ID)

def image_path(fig_id):
    return os.path.join(FIGURE_ID, fig_id)

def data_path(dat_id):
    return os.path.join(DATA_ID, dat_id)

def save_fig(fig_id):
    plt.savefig(image_path(fig_id) + ".png", format='png')

infile = open(data_path("EoS.csv"),'r')

# Read the EoS data as  csv file and organize the data into two arrays with density and energies
EoS = pd.read_csv(infile, names=('Density', 'Energy'))
EoS['Energy'] = pd.to_numeric(EoS['Energy'], errors='coerce')
EoS = EoS.dropna()
Energies = EoS['Energy']
Density = EoS['Density']
#  The design matrix now as function of various polytrops

Maxpolydegree = 30
X = np.zeros((len(Density),Maxpolydegree))
X[:,0] = 1.0
testerror = np.zeros(Maxpolydegree)
trainingerror = np.zeros(Maxpolydegree)
polynomial = np.zeros(Maxpolydegree)

trials = 100
for polydegree in range(1, Maxpolydegree):
    polynomial[polydegree] = polydegree
    for degree in range(polydegree):
        X[:,degree] = Density**(degree/3.0)

# loop over trials in order to estimate the expectation value of the MSE
    testerror[polydegree] = 0.0
    trainingerror[polydegree] = 0.0
    for samples in range(trials):
        x_train, x_test, y_train, y_test = train_test_split(X, Energies, test_size=0.2)
        model = LinearRegression(fit_intercept=True).fit(x_train, y_train)
        ypred = model.predict(x_train)
        ytilde = model.predict(x_test)
        testerror[polydegree] += mean_squared_error(y_test, ytilde)
        trainingerror[polydegree] += mean_squared_error(y_train, ypred) 

    testerror[polydegree] /= trials
    trainingerror[polydegree] /= trials

plt.plot(polynomial, np.log10(trainingerror), label='Training Error')
plt.plot(polynomial, np.log10(testerror), label='Test Error')
plt.xlabel('Polynomial degree')
plt.ylabel('log10[MSE]')
plt.legend()
plt.show()

\end{minted}




% !split
\subsection*{Resampling methods: Jackknife and Bootstrap}

Two famous
resampling methods are the \textbf{independent bootstrap} and \textbf{the jackknife}. 

The jackknife is a special case of the independent bootstrap. Still, the jackknife was made
popular prior to the independent bootstrap. And as the popularity of
the independent bootstrap soared, new variants, such as \textbf{the dependent bootstrap}.

The Jackknife and independent bootstrap work for
independent, identically distributed random variables.
If these conditions are not
satisfied, the methods will fail.


% !split
\subsection*{Resampling methods: Bootstrap}

% --- begin paragraph admon ---
\paragraph{}
Bootstrapping is a nonparametric approach to statistical inference
that substitutes computation for more traditional distributional
assumptions and asymptotic results. Bootstrapping offers a number of
advantages: 
\begin{enumerate}
\item The bootstrap is quite general, although there are some cases in which it fails.  

\item Because it does not require distributional assumptions (such as normally distributed errors), the bootstrap can provide more accurate inferences when the data are not well behaved or when the sample size is small.  

\item It is possible to apply the bootstrap to statistics with sampling distributions that are difficult to derive, even asymptotically. 

\item It is relatively simple to apply the bootstrap.
\end{enumerate}

\noindent
% --- end paragraph admon ---



% !split
\subsection*{Resampling methods: Bootstrap steps}

The independent bootstrap works like this: 

\begin{enumerate}
\item Draw with replacement $n$ numbers for the observed variables $\bm{x} = (x_1,x_2,\cdots,x_n)$. 

\item Define a vector $\bm{x}^*$ containing the values which were drawn from $\bm{x}$. 

\item Using the vector $\bm{x}^*$ compute the estimate (parameter) $\widehat{\theta}^*$ by evaluating $\widehat \theta$ under the observations $\bm{x}^*$. 

\item Repeat this process $k$ times. 
\end{enumerate}

\noindent
When you are done, you can draw a histogram of the relative frequency
of the estimator/parameter $\widehat \theta^*$. This is your estimate of the probability
distribution $p(x)$. Using this probability distribution you can
estimate any statistics thereof. In principle you never draw the
histogram of the relative frequency of $\widehat{\theta}^*$. Instead
you use the estimators corresponding to the statistic of interest. For
example, if you are interested in estimating the variance of $\widehat
\theta$, apply the etsimator $\widehat \sigma^2$ to the values
$\widehat \theta ^*$.


% !split
\subsection*{Code example for the Bootstrap method}

The following code starts with a Gaussian distribution with mean value
$\mu =100$ and variance $\sigma=15$. We use this to generate the data
used in the bootstrap analysis. The bootstrap analysis returns a data
set after a given number of bootstrap operations (as many as we have
data points). This data set consists of estimated mean values for each
bootstrap operation. The histogram generated by the bootstrap method
shows that the distribution for these mean values is also a Gaussian,
centered around the mean value $\mu=100$ but with standard deviation
$\sigma/\sqrt{n}$, where $n$ is the number of bootstrap samples (in
this case the same as the number of original data points). The value
of the standard deviation is what we expect from the central limit
theorem.


\begin{minted}[fontsize=\fontsize{9pt}{9pt},linenos=false,mathescape,baselinestretch=1.0,fontfamily=tt,xleftmargin=7mm]{python}
from numpy import *
from numpy.random import randint, randn
from time import time
import matplotlib.mlab as mlab
import matplotlib.pyplot as plt

# Returns mean of bootstrap samples                                                                                                                                                
def stat(data):
    return mean(data)

# Bootstrap algorithm
def bootstrap(data, statistic, R):
    t = zeros(R); n = len(data); inds = arange(n); t0 = time()
    # non-parametric bootstrap         
    for i in range(R):
        t[i] = statistic(data[randint(0,n,n)])

    # analysis    
    print("Runtime: %g sec" % (time()-t0)); print("Bootstrap Statistics :")
    print("original           bias      std. error")
    print("%8g %8g %14g %15g" % (statistic(data), std(data),mean(t),std(t)))
    return t


mu, sigma = 100, 15
datapoints = 10000
x = mu + sigma*random.randn(datapoints)
# bootstrap returns the data sample                                    
t = bootstrap(x, stat, datapoints)
# the histogram of the bootstrapped  data                                                                                                    
n, binsboot, patches = plt.hist(t, 50, normed=1, facecolor='red', alpha=0.75)

# add a 'best fit' line  
y = mlab.normpdf( binsboot, mean(t), std(t))
lt = plt.plot(binsboot, y, 'r--', linewidth=1)
plt.xlabel('Smarts')
plt.ylabel('Probability')
plt.axis([99.5, 100.6, 0, 3.0])
plt.grid(True)

plt.show()

\end{minted}


% !split 
\subsection*{Various steps in cross-validation}

When the repetitive splitting of the data set is done randomly,
samples may accidently end up in a fast majority of the splits in
either training or test set. Such samples may have an unbalanced
influence on either model building or prediction evaluation. To avoid
this $k$-fold cross-validation structures the data splitting. The
samples are divided into $k$ more or less equally sized exhaustive and
mutually exclusive subsets. In turn (at each split) one of these
subsets plays the role of the test set while the union of the
remaining subsets constitutes the training set. Such a splitting
warrants a balanced representation of each sample in both training and
test set over the splits. Still the division into the $k$ subsets
involves a degree of randomness. This may be fully excluded when
choosing $k=n$. This particular case is referred to as leave-one-out
cross-validation (LOOCV). 

% !split 
\subsection*{How to set up the cross-validation for Ridge and/or Lasso}

\begin{itemize}
\item Define a range of interest for the penalty parameter.

\item Divide the data set into training and test set comprising samples $\{1, \ldots, n\} \setminus i$ and $\{i \}$, respectively.

\item Fit the linear regression model by means of OLS, or Ridge or Lasso estimations  for each $\lambda$ in the grid using the training set, and the corresponding estimate of the error variance $\bm{\sigma}_{-i}^2(\lambda)$, as
\end{itemize}

\noindent
\begin{align*}
\bm{\beta}_{-i}(\lambda) & =  ( \bm{X}_{-i, \ast}^{T}
\bm{X}_{-i, \ast} + \lambda \bm{I}_{pp})^{-1}
\bm{X}_{-i, \ast}^{T} \bm{y}_{-i}
\end{align*}

\begin{itemize}
\item Evaluate the prediction performance of these models on the test set by $\log\{L[y_i, \bm{X}_{i, \ast}; \bm{\beta}_{-i}(\lambda), \bm{\sigma}_{-i}^2(\lambda)]\}$. Or, by the prediction error $|y_i - \bm{X}_{i, \ast} \bm{\beta}_{-i}(\lambda)|$, the relative error, the error squared or the R2 score function.

\item Repeat the first three steps  such that each sample plays the role of the test set once.

\item Average the prediction performances of the test sets at each grid point of the penalty bias/parameter. It is an estimate of the prediction performance of the model corresponding to this value of the penalty parameter on novel data. It is defined as
\end{itemize}

\noindent
\begin{align*}
\frac{1}{n} \sum_{i = 1}^n \log\{L[y_i, \mathbf{X}_{i, \ast}; \bm{\beta}_{-i}(\lambda), \bm{\sigma}_{-i}^2(\lambda)]\}.
\end{align*}

% !split
\subsection*{Cross-validation in brief}

For the various values of $k$

\begin{enumerate}
\item shuffle the dataset randomly.

\item Split the dataset into $k$ groups.

\item For each unique group:
\begin{enumerate}

\item Decide which group to use as set for test data

\item Take the remaining groups as a training data set

\item Fit a model on the training set and evaluate it on the test set

\item Retain the evaluation score and discard the model

\end{enumerate}

\noindent
\item Summarize the model using the sample of model evaluation scores
\end{enumerate}

\noindent
% !split
\subsection*{Code Example for Cross-validation and $k$-fold Cross-validation}

The code here uses Ridge regression with cross-validation (CV)  resampling and $k$-fold CV in order to fit a specific polynomial. 
\begin{minted}[fontsize=\fontsize{9pt}{9pt},linenos=false,mathescape,baselinestretch=1.0,fontfamily=tt,xleftmargin=7mm]{python}
import numpy as np
import matplotlib.pyplot as plt
from sklearn.model_selection import KFold
from sklearn.linear_model import Ridge
from sklearn.model_selection import cross_val_score
from sklearn.preprocessing import PolynomialFeatures

# A seed just to ensure that the random numbers are the same for every run.
# Useful for eventual debugging.
np.random.seed(3155)

# Generate the data.
nsamples = 100
x = np.random.randn(nsamples)
y = 3*x**2 + np.random.randn(nsamples)

## Cross-validation on Ridge regression using KFold only

# Decide degree on polynomial to fit
poly = PolynomialFeatures(degree = 6)

# Decide which values of lambda to use
nlambdas = 500
lambdas = np.logspace(-3, 5, nlambdas)

# Initialize a KFold instance
k = 5
kfold = KFold(n_splits = k)

# Perform the cross-validation to estimate MSE
scores_KFold = np.zeros((nlambdas, k))

i = 0
for lmb in lambdas:
    ridge = Ridge(alpha = lmb)
    j = 0
    for train_inds, test_inds in kfold.split(x):
        xtrain = x[train_inds]
        ytrain = y[train_inds]

        xtest = x[test_inds]
        ytest = y[test_inds]

        Xtrain = poly.fit_transform(xtrain[:, np.newaxis])
        ridge.fit(Xtrain, ytrain[:, np.newaxis])

        Xtest = poly.fit_transform(xtest[:, np.newaxis])
        ypred = ridge.predict(Xtest)

        scores_KFold[i,j] = np.sum((ypred - ytest[:, np.newaxis])**2)/np.size(ypred)

        j += 1
    i += 1


estimated_mse_KFold = np.mean(scores_KFold, axis = 1)

## Cross-validation using cross_val_score from sklearn along with KFold

# kfold is an instance initialized above as:
# kfold = KFold(n_splits = k)

estimated_mse_sklearn = np.zeros(nlambdas)
i = 0
for lmb in lambdas:
    ridge = Ridge(alpha = lmb)

    X = poly.fit_transform(x[:, np.newaxis])
    estimated_mse_folds = cross_val_score(ridge, X, y[:, np.newaxis], scoring='neg_mean_squared_error', cv=kfold)

    # cross_val_score return an array containing the estimated negative mse for every fold.
    # we have to the the mean of every array in order to get an estimate of the mse of the model
    estimated_mse_sklearn[i] = np.mean(-estimated_mse_folds)

    i += 1

## Plot and compare the slightly different ways to perform cross-validation

plt.figure()

plt.plot(np.log10(lambdas), estimated_mse_sklearn, label = 'cross_val_score')
plt.plot(np.log10(lambdas), estimated_mse_KFold, 'r--', label = 'KFold')

plt.xlabel('log10(lambda)')
plt.ylabel('mse')

plt.legend()

plt.show()

\end{minted}

% !split
\subsection*{How to decide upon the best model?}

% !split
\subsection*{The bias-variance tradeoff}


We will discuss the bias-variance tradeoff in the context of
continuous predictions such as regression. However, many of the
intuitions and ideas discussed here also carry over to classification
tasks. Consider a dataset $\mathcal{L}$ consisting of the data
$\mathbf{X}_\mathcal{L}=\{(y_j, \boldsymbol{x}_j), j=0\ldots n-1\}$. 

Let us assume that the true data is generated from a noisy model

\[
\bm{y}=f(\boldsymbol{x}) + \bm{\epsilon}
\]

where $\epsilon$ is normally distributed with mean zero and standard deviation $\sigma^2$.

In our derivation of the ordinary least squares method we defined then
an approximation to the function $f$ in terms of the parameters
$\bm{\beta}$ and the design matrix $\bm{X}$ which embody our model,
that is $\bm{\tilde{y}}=\bm{X}\bm{\beta}$. 

Thereafter we found the parameters $\bm{\beta}$ by optimizing the mean-squared error via the so-called cost function
\[
C(\bm{X},\bm{\beta}) =\frac{1}{n}\sum_{i=0}^{n-1}(y_i-\tilde{y}_i)^2=\mathbb{E}\left[(\bm{y}-\bm{\tilde{y}})^2\right].
\]

We can rewrite this as 
\[
\mathbb{E}\left[(\bm{y}-\bm{\tilde{y}})^2\right]=\frac{1}{n}\sum_i(f_i-\mathbb{E}\left[\bm{\tilde{y}}\right])^2+\frac{1}{n}\sum_i(\tilde{y}_i-\mathbb{E}\left[\bm{\tilde{y}}\right])^2+\sigma^2.
\]

The three terms represent the square of the bias of the learning
method, which can be thought of as the error caused by the simplifying
assumptions built into the method. The second term represents the
variance of the chosen model and finally the last terms is variance of
the error $\bm{\epsilon}$.

To derive this equation, we need to recall that the variance of $\bm{y}$ and $\bm{\epsilon}$ are both equal to $\sigma^2$. The mean value of $\bm{\epsilon}$ is by definition equal to zero. Furthermore, the function $f$ is not a stochastics variable, idem for $\bm{\tilde{y}}$.
We use a more compact notation in terms of the expectation value 
\[
\mathbb{E}\left[(\bm{y}-\bm{\tilde{y}})^2\right]=\mathbb{E}\left[(\bm{f}+\bm{\epsilon}-\bm{\tilde{y}})^2\right],
\]
and adding and subtracting $\mathbb{E}\left[\bm{\tilde{y}}\right]$ we get
\[
\mathbb{E}\left[(\bm{y}-\bm{\tilde{y}})^2\right]=\mathbb{E}\left[(\bm{f}+\bm{\epsilon}-\bm{\tilde{y}}+\mathbb{E}\left[\bm{\tilde{y}}\right]-\mathbb{E}\left[\bm{\tilde{y}}\right])^2\right],
\]
which, using the abovementioned expectation values can be rewritten as 
\[
\mathbb{E}\left[(\bm{y}-\bm{\tilde{y}})^2\right]=\mathbb{E}\left[(\bm{y}-\mathbb{E}\left[\bm{\tilde{y}}\right])^2\right]+\mathrm{Var}\left[\bm{\tilde{y}}\right]+\sigma^2,
\]
that is the rewriting in terms of the so-called bias, the variance of the model $\bm{\tilde{y}}$ and the variance of $\bm{\epsilon}$.



% !split
\subsection*{Understanding what happens}
\begin{minted}[fontsize=\fontsize{9pt}{9pt},linenos=false,mathescape,baselinestretch=1.0,fontfamily=tt,xleftmargin=7mm]{python}
import matplotlib.pyplot as plt
import numpy as np
from sklearn.linear_model import LinearRegression, Ridge, Lasso
from sklearn.preprocessing import PolynomialFeatures
from sklearn.model_selection import train_test_split
from sklearn.pipeline import make_pipeline
from sklearn.utils import resample

np.random.seed(2018)

n = 40
n_boostraps = 100
maxdegree = 14


# Make data set.
x = np.linspace(-3, 3, n).reshape(-1, 1)
y = np.exp(-x**2) + 1.5 * np.exp(-(x-2)**2)+ np.random.normal(0, 0.1, x.shape)
error = np.zeros(maxdegree)
bias = np.zeros(maxdegree)
variance = np.zeros(maxdegree)
polydegree = np.zeros(maxdegree)
x_train, x_test, y_train, y_test = train_test_split(x, y, test_size=0.2)

for degree in range(maxdegree):
    model = make_pipeline(PolynomialFeatures(degree=degree), LinearRegression(fit_intercept=False))
    y_pred = np.empty((y_test.shape[0], n_boostraps))
    for i in range(n_boostraps):
        x_, y_ = resample(x_train, y_train)
        y_pred[:, i] = model.fit(x_, y_).predict(x_test).ravel()

    polydegree[degree] = degree
    error[degree] = np.mean( np.mean((y_test - y_pred)**2, axis=1, keepdims=True) )
    bias[degree] = np.mean( (y_test - np.mean(y_pred, axis=1, keepdims=True))**2 )
    variance[degree] = np.mean( np.var(y_pred, axis=1, keepdims=True) )
#    print('Polynomial degree:', degree)
#    print('Error:', error[degree])
#    print('Bias^2:', bias[degree])
#    print('Var:', variance[degree])
#    print('{} >= {} + {} = {}'.format(error[degree], bias[degree], variance[degree], bias[degree]+variance[degree]))

plt.plot(polydegree, error, label='Error')
plt.plot(polydegree, bias, label='bias')
plt.plot(polydegree, variance, label='Variance')
plt.legend()
plt.show()




\end{minted}

% !split 
\subsection*{Summing up}




The bias-variance tradeoff summarizes the fundamental tension in
machine learning, particularly supervised learning, between the
complexity of a model and the amount of training data needed to train
it.  Since data is often limited, in practice it is often useful to
use a less-complex model with higher bias, that is  a model whose asymptotic
performance is worse than another model because it is easier to
train and less sensitive to sampling noise arising from having a
finite-sized training dataset (smaller variance). 



The above equations tell us that in
order to minimize the expected test error, we need to select a
statistical learning method that simultaneously achieves low variance
and low bias. Note that variance is inherently a nonnegative quantity,
and squared bias is also nonnegative. Hence, we see that the expected
test MSE can never lie below $Var(\epsilon)$, the irreducible error.


What do we mean by the variance and bias of a statistical learning
method? The variance refers to the amount by which our model would change if we
estimated it using a different training data set. Since the training
data are used to fit the statistical learning method, different
training data sets  will result in a different estimate. But ideally the
estimate for our model should not vary too much between training
sets. However, if a method has high variance  then small changes in
the training data can result in large changes in the model. In general, more
flexible statistical methods have higher variance.


You may also find this recent \href{{https://www.pnas.org/content/116/32/15849}}{article} of interest.



% !split 
\subsection*{The same example but now with cross-validation}

\begin{minted}[fontsize=\fontsize{9pt}{9pt},linenos=false,mathescape,baselinestretch=1.0,fontfamily=tt,xleftmargin=7mm]{python}
# Common imports
import os
import numpy as np
import pandas as pd
import matplotlib.pyplot as plt
from sklearn.linear_model import LinearRegression, Ridge, Lasso
from sklearn.metrics import mean_squared_error
from sklearn.model_selection import KFold
from sklearn.model_selection import cross_val_score


# Where to save the figures and data files
PROJECT_ROOT_DIR = "Results"
FIGURE_ID = "Results/FigureFiles"
DATA_ID = "DataFiles/"

if not os.path.exists(PROJECT_ROOT_DIR):
    os.mkdir(PROJECT_ROOT_DIR)

if not os.path.exists(FIGURE_ID):
    os.makedirs(FIGURE_ID)

if not os.path.exists(DATA_ID):
    os.makedirs(DATA_ID)

def image_path(fig_id):
    return os.path.join(FIGURE_ID, fig_id)

def data_path(dat_id):
    return os.path.join(DATA_ID, dat_id)

def save_fig(fig_id):
    plt.savefig(image_path(fig_id) + ".png", format='png')

infile = open(data_path("EoS.csv"),'r')

# Read the EoS data as  csv file and organize the data into two arrays with density and energies
EoS = pd.read_csv(infile, names=('Density', 'Energy'))
EoS['Energy'] = pd.to_numeric(EoS['Energy'], errors='coerce')
EoS = EoS.dropna()
Energies = EoS['Energy']
Density = EoS['Density']
#  The design matrix now as function of various polytrops

Maxpolydegree = 30
X = np.zeros((len(Density),Maxpolydegree))
X[:,0] = 1.0
estimated_mse_sklearn = np.zeros(Maxpolydegree)
polynomial = np.zeros(Maxpolydegree)
k =5
kfold = KFold(n_splits = k)

for polydegree in range(1, Maxpolydegree):
    polynomial[polydegree] = polydegree
    for degree in range(polydegree):
        X[:,degree] = Density**(degree/3.0)
        OLS = LinearRegression()
# loop over trials in order to estimate the expectation value of the MSE
    estimated_mse_folds = cross_val_score(OLS, X, Energies, scoring='neg_mean_squared_error', cv=kfold)
#[:, np.newaxis]
    estimated_mse_sklearn[polydegree] = np.mean(-estimated_mse_folds)

plt.plot(polynomial, np.log10(estimated_mse_sklearn), label='Test Error')
plt.xlabel('Polynomial degree')
plt.ylabel('log10[MSE]')
plt.legend()
plt.show()

\end{minted}

% !split
\subsection*{Cross-validation with Ridge}
\begin{minted}[fontsize=\fontsize{9pt}{9pt},linenos=false,mathescape,baselinestretch=1.0,fontfamily=tt,xleftmargin=7mm]{python}
import numpy as np
import matplotlib.pyplot as plt
from sklearn.model_selection import KFold
from sklearn.linear_model import Ridge
from sklearn.model_selection import cross_val_score
from sklearn.preprocessing import PolynomialFeatures

# A seed just to ensure that the random numbers are the same for every run.
np.random.seed(3155)
# Generate the data.
n = 100
x = np.linspace(-3, 3, n).reshape(-1, 1)
y = np.exp(-x**2) + 1.5 * np.exp(-(x-2)**2)+ np.random.normal(0, 0.1, x.shape)
# Decide degree on polynomial to fit
poly = PolynomialFeatures(degree = 10)

# Decide which values of lambda to use
nlambdas = 500
lambdas = np.logspace(-3, 5, nlambdas)
# Initialize a KFold instance
k = 5
kfold = KFold(n_splits = k)
estimated_mse_sklearn = np.zeros(nlambdas)
i = 0
for lmb in lambdas:
    ridge = Ridge(alpha = lmb)
    estimated_mse_folds = cross_val_score(ridge, x, y, scoring='neg_mean_squared_error', cv=kfold)
    estimated_mse_sklearn[i] = np.mean(-estimated_mse_folds)
    i += 1
plt.figure()
plt.plot(np.log10(lambdas), estimated_mse_sklearn, label = 'cross_val_score')
plt.xlabel('log10(lambda)')
plt.ylabel('MSE')
plt.legend()
plt.show()


\end{minted}






% !split
\subsection*{Applying Regression Analysis to the Ising Model}



% !split
\subsection*{The Ising model}

The one-dimensional Ising model with nearest neighbor interaction, no
external field and a constant coupling constant $J$ is given by

\begin{align}
    H = -J \sum_{k}^L s_k s_{k + 1},
\end{align}

where $s_i \in \{-1, 1\}$ and $s_{N + 1} = s_1$. The number of spins
in the system is determined by $L$. For the one-dimensional system
there is no phase transition.

We will look at a system of $L = 40$ spins with a coupling constant of
$J = 1$. To get enough training data we will generate 10000 states
with their respective energies.


\begin{minted}[fontsize=\fontsize{9pt}{9pt},linenos=false,mathescape,baselinestretch=1.0,fontfamily=tt,xleftmargin=7mm]{python}
import numpy as np
import matplotlib.pyplot as plt
from mpl_toolkits.axes_grid1 import make_axes_locatable
import seaborn as sns
import scipy.linalg as scl
from sklearn.model_selection import train_test_split
import tqdm
sns.set(color_codes=True)
cmap_args=dict(vmin=-1., vmax=1., cmap='seismic')

L = 40
n = int(1e4)

spins = np.random.choice([-1, 1], size=(n, L))
J = 1.0

energies = np.zeros(n)

for i in range(n):
    energies[i] = - J * np.dot(spins[i], np.roll(spins[i], 1))
\end{minted}

Here we use ordinary least squares
regression to predict the energy for the nearest neighbor
one-dimensional Ising model on a ring, i.e., the endpoints wrap
around. We will use linear regression to fit a value for
the coupling constant to achieve this.

% !split
\subsection*{Reformulating the problem to suit regression}

A more general form for the one-dimensional Ising model is

\begin{align}
    H = - \sum_j^L \sum_k^L s_j s_k J_{jk}.
\end{align}

Here we allow for interactions beyond the nearest neighbors and a state dependent
coupling constant. This latter expression can be formulated as
a matrix-product
\begin{align}
    \bm{H} = \bm{X} J,
\end{align}

where $X_{jk} = s_j s_k$ and $J$ is a matrix which consists of the
elements $-J_{jk}$. This form of writing the energy fits perfectly
with the form utilized in linear regression, that is

\begin{align}
    \bm{y} = \bm{X}\bm{\beta} + \bm{\epsilon},
\end{align}

We split the data in training and test data as discussed in the previous example

\begin{minted}[fontsize=\fontsize{9pt}{9pt},linenos=false,mathescape,baselinestretch=1.0,fontfamily=tt,xleftmargin=7mm]{python}
X = np.zeros((n, L ** 2))
for i in range(n):
    X[i] = np.outer(spins[i], spins[i]).ravel()
y = energies
X_train, X_test, y_train, y_test = train_test_split(X, y, test_size=0.2)
\end{minted}

% !split
\subsection*{Linear regression}

In the ordinary least squares method we choose the cost function

\begin{align}
    C(\bm{X}, \bm{\beta})= \frac{1}{n}\left\{(\bm{X}\bm{\beta} - \bm{y})^T(\bm{X}\bm{\beta} - \bm{y})\right\}.
\end{align}

We then find the extremal point of $C$ by taking the derivative with respect to $\bm{\beta}$ as discussed above.
This yields the expression for $\bm{\beta}$ to be

\[
    \bm{\beta} = \frac{\bm{X}^T \bm{y}}{\bm{X}^T \bm{X}},
\]

which immediately imposes some requirements on $\bm{X}$ as there must exist
an inverse of $\bm{X}^T \bm{X}$. If the expression we are modeling contains an
intercept, i.e., a constant term, we must make sure that the
first column of $\bm{X}$ consists of $1$. We do this here

\begin{minted}[fontsize=\fontsize{9pt}{9pt},linenos=false,mathescape,baselinestretch=1.0,fontfamily=tt,xleftmargin=7mm]{python}
X_train_own = np.concatenate(
    (np.ones(len(X_train))[:, np.newaxis], X_train),
    axis=1
)
X_test_own = np.concatenate(
    (np.ones(len(X_test))[:, np.newaxis], X_test),
    axis=1
)
\end{minted}

\begin{minted}[fontsize=\fontsize{9pt}{9pt},linenos=false,mathescape,baselinestretch=1.0,fontfamily=tt,xleftmargin=7mm]{python}
def ols_inv(x: np.ndarray, y: np.ndarray) -> np.ndarray:
    return scl.inv(x.T @ x) @ (x.T @ y)
beta = ols_inv(X_train_own, y_train)
\end{minted}


% !split
\subsection*{Singular Value decomposition}

Doing the inversion directly turns out to be a bad idea since the matrix
$\bm{X}^T\bm{X}$ is singular. An alternative approach is to use the \textbf{singular
value decomposition}. Using the definition of the Moore-Penrose
pseudoinverse we can write the equation for $\bm{\beta}$ as

\[
    \bm{\beta} = \bm{X}^{+}\bm{y},
\]

where the pseudoinverse of $\bm{X}$ is given by

\[
    \bm{X}^{+} = \frac{\bm{X}^T}{\bm{X}^T\bm{X}}.
\]

Using singular value decomposition we can decompose the matrix  $\bm{X} = \bm{U}\bm{\Sigma} \bm{V}^T$,
where $\bm{U}$ and $\bm{V}$ are orthogonal(unitary) matrices and $\bm{\Sigma}$ contains the singular values (more details below).
where $X^{+} = V\Sigma^{+} U^T$. This reduces the equation for
$\omega$ to
\begin{align}
    \bm{\beta} = \bm{V}\bm{\Sigma}^{+} \bm{U}^T \bm{y}.
\end{align}

Note that solving this equation by actually doing the pseudoinverse
(which is what we will do) is not a good idea as this operation scales
as $\mathcal{O}(n^3)$, where $n$ is the number of elements in a
general matrix. Instead, doing $QR$-factorization and solving the
linear system as an equation would reduce this down to
$\mathcal{O}(n^2)$ operations.


\begin{minted}[fontsize=\fontsize{9pt}{9pt},linenos=false,mathescape,baselinestretch=1.0,fontfamily=tt,xleftmargin=7mm]{python}
def ols_svd(x: np.ndarray, y: np.ndarray) -> np.ndarray:
    u, s, v = scl.svd(x)
    return v.T @ scl.pinv(scl.diagsvd(s, u.shape[0], v.shape[0])) @ u.T @ y
\end{minted}

\begin{minted}[fontsize=\fontsize{9pt}{9pt},linenos=false,mathescape,baselinestretch=1.0,fontfamily=tt,xleftmargin=7mm]{python}
beta = ols_svd(X_train_own,y_train)
\end{minted}

When extracting the $J$-matrix  we need to make sure that we remove the intercept, as is done here

\begin{minted}[fontsize=\fontsize{9pt}{9pt},linenos=false,mathescape,baselinestretch=1.0,fontfamily=tt,xleftmargin=7mm]{python}
J = beta[1:].reshape(L, L)
\end{minted}

A way of looking at the coefficients in $J$ is to plot the matrices as images.


\begin{minted}[fontsize=\fontsize{9pt}{9pt},linenos=false,mathescape,baselinestretch=1.0,fontfamily=tt,xleftmargin=7mm]{python}
fig = plt.figure(figsize=(20, 14))
im = plt.imshow(J, **cmap_args)
plt.title("OLS", fontsize=18)
plt.xticks(fontsize=18)
plt.yticks(fontsize=18)
cb = fig.colorbar(im)
cb.ax.set_yticklabels(cb.ax.get_yticklabels(), fontsize=18)
plt.show()
\end{minted}
It is interesting to note that OLS
considers both $J_{j, j + 1} = -0.5$ and $J_{j, j - 1} = -0.5$ as
valid matrix elements for $J$.
In our discussion below on hyperparameters and Ridge and Lasso regression we will see that
this problem can be removed, partly and only with Lasso regression. 

In this case our matrix inversion was actually possible. The obvious question now is what is the mathematics behind the SVD?





% !split
\subsection*{The one-dimensional Ising model}

Let us bring back the Ising model again, but now with an additional
focus on Ridge and Lasso regression as well. We repeat some of the
basic parts of the Ising model and the setup of the training and test
data.  The one-dimensional Ising model with nearest neighbor
interaction, no external field and a constant coupling constant $J$ is
given by

\begin{align}
    H = -J \sum_{k}^L s_k s_{k + 1},
\end{align}
where $s_i \in \{-1, 1\}$ and $s_{N + 1} = s_1$. The number of spins in the system is determined by $L$. For the one-dimensional system there is no phase transition.

We will look at a system of $L = 40$ spins with a coupling constant of $J = 1$. To get enough training data we will generate 10000 states with their respective energies.


\begin{minted}[fontsize=\fontsize{9pt}{9pt},linenos=false,mathescape,baselinestretch=1.0,fontfamily=tt,xleftmargin=7mm]{python}
import numpy as np
import matplotlib.pyplot as plt
from mpl_toolkits.axes_grid1 import make_axes_locatable
import seaborn as sns
import scipy.linalg as scl
from sklearn.model_selection import train_test_split
import sklearn.linear_model as skl
import tqdm
sns.set(color_codes=True)
cmap_args=dict(vmin=-1., vmax=1., cmap='seismic')

L = 40
n = int(1e4)

spins = np.random.choice([-1, 1], size=(n, L))
J = 1.0

energies = np.zeros(n)

for i in range(n):
    energies[i] = - J * np.dot(spins[i], np.roll(spins[i], 1))
\end{minted}

A more general form for the one-dimensional Ising model is

\begin{align}
    H = - \sum_j^L \sum_k^L s_j s_k J_{jk}.
\end{align}

Here we allow for interactions beyond the nearest neighbors and a more
adaptive coupling matrix. This latter expression can be formulated as
a matrix-product on the form
\begin{align}
    H = X J,
\end{align}

where $X_{jk} = s_j s_k$ and $J$ is the matrix consisting of the
elements $-J_{jk}$. This form of writing the energy fits perfectly
with the form utilized in linear regression, viz.
\begin{align}
    \bm{y} = \bm{X}\bm{\beta} + \bm{\epsilon}.
\end{align}
We organize the data as we did above
\begin{minted}[fontsize=\fontsize{9pt}{9pt},linenos=false,mathescape,baselinestretch=1.0,fontfamily=tt,xleftmargin=7mm]{python}
X = np.zeros((n, L ** 2))
for i in range(n):
    X[i] = np.outer(spins[i], spins[i]).ravel()
y = energies
X_train, X_test, y_train, y_test = train_test_split(X, y, test_size=0.96)

X_train_own = np.concatenate(
    (np.ones(len(X_train))[:, np.newaxis], X_train),
    axis=1
)

X_test_own = np.concatenate(
    (np.ones(len(X_test))[:, np.newaxis], X_test),
    axis=1
)
\end{minted}

We will do all fitting with \textbf{Scikit-Learn},

\begin{minted}[fontsize=\fontsize{9pt}{9pt},linenos=false,mathescape,baselinestretch=1.0,fontfamily=tt,xleftmargin=7mm]{python}
clf = skl.LinearRegression().fit(X_train, y_train)
\end{minted}
When  extracting the $J$-matrix we make sure to remove the intercept
\begin{minted}[fontsize=\fontsize{9pt}{9pt},linenos=false,mathescape,baselinestretch=1.0,fontfamily=tt,xleftmargin=7mm]{python}
J_sk = clf.coef_.reshape(L, L)
\end{minted}
And then we plot the results
\begin{minted}[fontsize=\fontsize{9pt}{9pt},linenos=false,mathescape,baselinestretch=1.0,fontfamily=tt,xleftmargin=7mm]{python}
fig = plt.figure(figsize=(20, 14))
im = plt.imshow(J_sk, **cmap_args)
plt.title("LinearRegression from Scikit-learn", fontsize=18)
plt.xticks(fontsize=18)
plt.yticks(fontsize=18)
cb = fig.colorbar(im)
cb.ax.set_yticklabels(cb.ax.get_yticklabels(), fontsize=18)
plt.show()
\end{minted}
The results perfectly with our previous discussion where we used our own code.

% !split
\subsection*{Ridge regression}

Having explored the ordinary least squares we move on to ridge
regression. In ridge regression we include a \textbf{regularizer}. This
involves a new cost function which leads to a new estimate for the
weights $\bm{\beta}$. This results in a penalized regression problem. The
cost function is given by

\begin{align}
    C(\bm{X}, \bm{\beta}; \lambda) = (\bm{X}\bm{\beta} - \bm{y})^T(\bm{X}\bm{\beta} - \bm{y}) + \lambda \bm{\beta}^T\bm{\beta}.
\end{align}
\begin{minted}[fontsize=\fontsize{9pt}{9pt},linenos=false,mathescape,baselinestretch=1.0,fontfamily=tt,xleftmargin=7mm]{python}
_lambda = 0.1
clf_ridge = skl.Ridge(alpha=_lambda).fit(X_train, y_train)
J_ridge_sk = clf_ridge.coef_.reshape(L, L)
fig = plt.figure(figsize=(20, 14))
im = plt.imshow(J_ridge_sk, **cmap_args)
plt.title("Ridge from Scikit-learn", fontsize=18)
plt.xticks(fontsize=18)
plt.yticks(fontsize=18)
cb = fig.colorbar(im)
cb.ax.set_yticklabels(cb.ax.get_yticklabels(), fontsize=18)

plt.show()
\end{minted}

% !split
\subsection*{LASSO regression}

In the \textbf{Least Absolute Shrinkage and Selection Operator} (LASSO)-method we get a third cost function.

\begin{align}
    C(\bm{X}, \bm{\beta}; \lambda) = (\bm{X}\bm{\beta} - \bm{y})^T(\bm{X}\bm{\beta} - \bm{y}) + \lambda \sqrt{\bm{\beta}^T\bm{\beta}}.
\end{align}

Finding the extremal point of this cost function is not so straight-forward as in least squares and ridge. We will therefore rely solely on the function ``Lasso`` from \textbf{Scikit-Learn}.

\begin{minted}[fontsize=\fontsize{9pt}{9pt},linenos=false,mathescape,baselinestretch=1.0,fontfamily=tt,xleftmargin=7mm]{python}
clf_lasso = skl.Lasso(alpha=_lambda).fit(X_train, y_train)
J_lasso_sk = clf_lasso.coef_.reshape(L, L)
fig = plt.figure(figsize=(20, 14))
im = plt.imshow(J_lasso_sk, **cmap_args)
plt.title("Lasso from Scikit-learn", fontsize=18)
plt.xticks(fontsize=18)
plt.yticks(fontsize=18)
cb = fig.colorbar(im)
cb.ax.set_yticklabels(cb.ax.get_yticklabels(), fontsize=18)

plt.show()
\end{minted}

It is quite striking how LASSO breaks the symmetry of the coupling
constant as opposed to ridge and OLS. We get a sparse solution with
$J_{j, j + 1} = -1$.



% !split
\subsection*{Performance as  function of the regularization parameter}

We see how the different models perform for a different set of values for $\lambda$.


\begin{minted}[fontsize=\fontsize{9pt}{9pt},linenos=false,mathescape,baselinestretch=1.0,fontfamily=tt,xleftmargin=7mm]{python}
lambdas = np.logspace(-4, 5, 10)

train_errors = {
    "ols_sk": np.zeros(lambdas.size),
    "ridge_sk": np.zeros(lambdas.size),
    "lasso_sk": np.zeros(lambdas.size)
}

test_errors = {
    "ols_sk": np.zeros(lambdas.size),
    "ridge_sk": np.zeros(lambdas.size),
    "lasso_sk": np.zeros(lambdas.size)
}

plot_counter = 1

fig = plt.figure(figsize=(32, 54))

for i, _lambda in enumerate(tqdm.tqdm(lambdas)):
    for key, method in zip(
        ["ols_sk", "ridge_sk", "lasso_sk"],
        [skl.LinearRegression(), skl.Ridge(alpha=_lambda), skl.Lasso(alpha=_lambda)]
    ):
        method = method.fit(X_train, y_train)

        train_errors[key][i] = method.score(X_train, y_train)
        test_errors[key][i] = method.score(X_test, y_test)

        omega = method.coef_.reshape(L, L)

        plt.subplot(10, 5, plot_counter)
        plt.imshow(omega, **cmap_args)
        plt.title(r"%s, $\lambda = %.4f$" % (key, _lambda))
        plot_counter += 1

plt.show()
\end{minted}

We see that LASSO reaches a good solution for low
values of $\lambda$, but will "wither" when we increase $\lambda$ too
much. Ridge is more stable over a larger range of values for
$\lambda$, but eventually also fades away.

% !split
\subsection*{Finding the optimal value of $\lambda$}

To determine which value of $\lambda$ is best we plot the accuracy of
the models when predicting the training and the testing set. We expect
the accuracy of the training set to be quite good, but if the accuracy
of the testing set is much lower this tells us that we might be
subject to an overfit model. The ideal scenario is an accuracy on the
testing set that is close to the accuracy of the training set.


\begin{minted}[fontsize=\fontsize{9pt}{9pt},linenos=false,mathescape,baselinestretch=1.0,fontfamily=tt,xleftmargin=7mm]{python}
fig = plt.figure(figsize=(20, 14))

colors = {
    "ols_sk": "r",
    "ridge_sk": "y",
    "lasso_sk": "c"
}

for key in train_errors:
    plt.semilogx(
        lambdas,
        train_errors[key],
        colors[key],
        label="Train {0}".format(key),
        linewidth=4.0
    )

for key in test_errors:
    plt.semilogx(
        lambdas,
        test_errors[key],
        colors[key] + "--",
        label="Test {0}".format(key),
        linewidth=4.0
    )
plt.legend(loc="best", fontsize=18)
plt.xlabel(r"$\lambda$", fontsize=18)
plt.ylabel(r"$R^2$", fontsize=18)
plt.tick_params(labelsize=18)
plt.show()
\end{minted}

From the above figure we can see that LASSO with $\lambda = 10^{-2}$
achieves a very good accuracy on the test set. This by far surpasses the
other models for all values of $\lambda$.





% ------------------- end of main content ---------------

\end{document}

